\documentclass{crortiz_cv_2021}

\setname{Carlos}{Ortiz-Gómez}
\setmobile{+1 857 472 1960}
\setmail{orgoca@gmail.com}
\setposition{Data Scientist} %ignored for now
\setlinkedinaccount{https://www.linkedin.com/orgoca} %you can play with color of the template (red is also nice..)
\setgithubaccount{https://github.com/orgoca} %you can play with color of the template (red is also nice..)
\definecolor{udemy}{RGB}{235, 82, 79}
\setthemecolor{udemy} %you can play with color of the template (red is also nice..)

\begin{document}
%Set variables
%You can add sections, texts, explanations just by copying the style below. Replace the dummy texts "\lipsum[1][x-x]\par" with actual texts.
%Create header
\headerview
\vspace{1ex}
%Sections
%
\setcompanyname{Bain \& Company}
\setcontactperson{Miembros del 'Advanced Analytics Group', Bain \& Company}
\setclaimedposition{Manager, Data Science} 
\coverletter{ %

\indent \ Es con un gusto enorme que presento mi candidatura para ocupar la posición de Manager, Data Science en su grupo de trabajo. Mi nombre es Carlos Ortiz, y soy un profesional con 15 años de experiencia demostrando resultados en el sector público, privado, y del emprendimiento. 
\vspace{5mm} %5mm vertical space

\indent \ Me apasiona resolver problemas y aplicar nuevas herramientas para ser más efectivo al hacerlo. He explorado a profundidad el poder de los datos para descubrir patrones y perspectivas que de otra manera no son evidentes. Mi visión de los datos se basa en la idea de que cualquier profesional, independientemente de su giro, debe ser capaz de tomar mejores decisiones basadas en datos y evidencia. Mi aspiración como profesional, es habilitar las condiciones para qué las organizaciones con las que colaboro mejoren su capacidad de toma de decisiones basadas en datos. Esto no con un modelo que centraliza el análisis de datos en un departamento de data science, sino en un modelo distribuido en el que cada rincón de la organización existen las capacidades para hacerlo. Para lograr esta ‘democratización de datos’, me he especializado en entender quiénes son mis usuarios, cuáles son los procesos y tomas de decisión en los que requieren de datos, y facilitarles los medios para que puedan lograrlo. Me motiva comunicar datos a audiencias no expertas, entender y automatizar el flujo de datos del punto de generación al punto de consumo, desarrollar analíticos, dashboards y templetes que destilen lo importante que comunican los datos. 
\vspace{5mm} %5mm vertical space

\indent \ La posición que buscan llenar requiere de competencias técnicas, pero también de una probada trayectoria gestionando equipos, creando y comunicando una visión común y atrayendo nuevos negocios a la organización. Mi trayectoria como Director General en la Secretaria de Energía y emprendedor de tres startups de tecnología, muestra qué además tengo todas estas capacidades. Como muestra, en la SENER logré movilizar al sector académico y industrial a agendas con visión compartida en temas de investigación y desarrollo de capacidades técnicas. Armé equipos desde cero e instrumentamos procesos, estándares y protocolos de comunicación que permitieron que una oficina con una cartera de proyectos de arriba de 1 billón de dólares funcionara de manera eficiente y transparente. Motivé y promoví el desarrollo académico y profesional de muchos jóvenes que se acercaron a nuestro proyecto y que de una manera u otra participaron en el. 

\vspace{5mm} %5mm vertical space
\indent \ Me motiva mucho la idea de aplicar conocimientos a la gran diversidad de clientes y retos que acuden a Bain \& Company para resolver. Cuando pienso en la mejor manera de continuar en mi viaje de aprendizaje y desarrollo profesional, me espanta la idea de entrar en la monotonía que rápidamente se puede generar cuando un puñado de trucos son los que resuelven la mayoría de los problemas que puedes enfrentar en tu vida laboral. En una organización como Bain \& Company, no solo existe diversidad de retos, sino que, además, los clientes que acuden a ustedes acuden con los retos que mas estresan las capacidades internas de su organización. Exactamente el tipo de retos que presentan los problemas que me motiva resolver.

\vspace{5mm} %5mm vertical space
\ Agradezco su tiempo,

\vspace{5mm} %5mm vertical space
\ Carlos Ortiz-Gómez}
%replace this part with actual text

%Footnote
\createfootnote
\end{document}